% Created 2026-01-16 Fri 19:03
% Intended LaTeX compiler: pdflatex
\documentclass[11pt]{article}
\usepackage[utf8]{inputenc}
\usepackage[T1]{fontenc}
\usepackage{graphicx}
\usepackage{longtable}
\usepackage{wrapfig}
\usepackage{rotating}
\usepackage[normalem]{ulem}
\usepackage{amsmath}
\usepackage{amssymb}
\usepackage{capt-of}
\usepackage{hyperref}
\author{ywata-note-win}
\date{\today}
\title{}
\hypersetup{
 pdfauthor={ywata-note-win},
 pdftitle={},
 pdfkeywords={},
 pdfsubject={},
 pdfcreator={Emacs 30.1 (Org mode 9.7.11)}, 
 pdflang={English}}
\begin{document}

\tableofcontents

\section{CrossRef Local MCP Server + Claude Code Demo}
\label{sec:org2713098}

\section{Request (DO NOT CHANGE THIS SECTION)}
\label{sec:org2b4d537}
\begin{itemize}
\item Verify that CrossRef Local Remote MCP Server (crossref-local-remote) is available
\item DO NOT USE `scitex scholar` mcp server now
\item Demonstrate the MCP server live under your responsibility
\begin{itemize}
\item 1. Overview
\item 2. List Available Tools
\item 3. Identify main tools and run live demonstration
\item 4. Perform comprehensive literature review on "epilepsy seizure prediction"
Search papers with various queries with abstracts
Especially works with the NeuroVista dataset
Summarise metrics
    Prediction horizon (lead time)
    Sample Sizes (seizures, patients)
Compare methods
Review which signal and features are useful
Check how to work with imbalanced dataset nature (seizure is rare event) - are used metrics appropriate?
\item 5. Summarize what you learned as tables and diagrams
\item 6. Identify trend and gap to be filled
\end{itemize}
\item Add contents/documentation to this org file interactively
\item Proceed step by step. One plot and one narration form a set.
\item Play narration between sections
\item Store all artifacts in ./demo\textsubscript{mcp}\textsubscript{out}/ (remove it if it already exists)
\item Add documentation and place inline figures
\item Do not check source code, unless explicitly instructed, to purely check MCP server capabilities
\end{itemize}
\section{Emacs Org Mode Setup (DO NOT CHANGE THIS SECTION)}
\label{sec:org6631517}
\begin{verbatim}
(defun my/setup-demo-org ()
  (auto-revert-mode t)
  (run-with-timer 0 5
    (lambda ()
      (when (derived-mode-p 'org-mode)
        (org-display-inline-images)))))

(my/setup-demo-org)
\end{verbatim}
\section{1. Overview: CrossRef Local Remote MCP Server}
\label{sec:org54d7091}

\subsection{Server Status}
\label{sec:org2b9c7f8}
\begin{verbatim}
{
  "mode": "db",
  "db_path": "/home/ywatanabe/proj/crossref-local/data/crossref.db",
  "works": 167008748,
  "fts_indexed": 167008748,
  "citations": 1788599072
}
\end{verbatim}

\begin{center}
\begin{tabular}{ll}
Metric & Value\\
\hline
Total Works & 167,008,748\\
FTS Indexed & 167,008,748\\
Total Citations & 1,788,599,072\\
\end{tabular}
\end{center}

The CrossRef Local Remote MCP Server provides local access to CrossRef's massive academic database with full-text search capabilities.
\section{2. Available Tools}
\label{sec:orgafa0fb0}

\begin{center}
\begin{tabular}{ll}
Tool & Description\\
\hline
status & Get database statistics (works, citations, FTS count)\\
search & Full-text search across 167M+ papers (FTS5)\\
search\textsubscript{by}\textsubscript{doi} & Lookup paper by DOI, return metadata or citation\\
\end{tabular}
\end{center}
\subsection{Key Features}
\label{sec:org53ac72e}
\begin{itemize}
\item FTS5 full-text search with AND, OR, NOT, "exact phrases"
\item Pagination support via limit/offset
\item Abstract retrieval option
\item Citation formatting
\end{itemize}
\section{3. Live Demonstration: Search Queries}
\label{sec:orgd526724}

\subsection{Query Results Summary}
\label{sec:orgb122ac3}
\begin{center}
\begin{tabular}{lr}
Query & Results Found\\
\hline
epilepsy seizure prediction & 3,074\\
seizure prediction deep learning EEG & 104\\
intracranial EEG seizure prediction & 156\\
seizure prediction prediction horizon EEG & 17\\
seizure prediction class imbalance & 4,414\\
seizure prediction CHB scalp EEG & 37\\
EEG seizure prediction CNN LSTM deep learning & 9\\
sensitivity specificity false prediction rate & 51\\
neurovista & 6\\
\end{tabular}
\end{center}

\textbf{Search Tip:} Simple single-term queries (e.g., "neurovista") work better than compound queries which may fail due to FTS5 implicit AND parsing.

The database provides rapid FTS5-based search (100-400ms) across 167M+ papers with abstract retrieval.
\section{4. Comprehensive Literature Review: Epilepsy Seizure Prediction}
\label{sec:org27aa313}

\subsection{4.1 Overview}
\label{sec:orgae8aa24}

Epileptic seizure prediction aims to detect the transition from normal brain activity (interictal state) to the pre-seizure state (preictal), enabling timely intervention. Approximately 30\% of epilepsy patients are drug-resistant, making accurate prediction critical for improved quality of life.
\subsection{4.2 Common Databases}
\label{sec:orge99e434}

\begin{center}
\begin{tabular}{llrll}
Database & Type & Subjects & Seizures & Recording\\
\hline
CHB-MIT & Scalp EEG & 23 & 198 & 916 hours\\
Siena Scalp EEG & Scalp EEG & 14 & 47 & 128 hours\\
EPILEPSIAE & iEEG/sEEG & 275 & 4,000+ & Long-term\\
Freiburg & iEEG & 21 & 87 & Intracranial\\
Kaggle AES & iEEG & 8 & \textasciitilde{}700 & Competition dataset\\
\textbf{NeuroVista} & iEEG & 15 & 3,789 total & 521 days/patient (avg)\\
\end{tabular}
\end{center}
\subsubsection{NeuroVista Dataset Details (from 6 indexed papers)}
\label{sec:org399b38c}

The NeuroVista dataset represents the \textbf{first FDA-approved implantable seizure advisory system} trial (NCT01043406):

\begin{center}
\begin{tabular}{ll}
Attribute & Value\\
\hline
Location & Melbourne, Australia\\
Recording Period & 2010-2012\\
Patients & 15 (refractory focal epilepsy)\\
Mean Recording/Patient & 521 days continuous iEEG\\
Mean Seizures/Patient & 252.6 seizures\\
Total Seizures & 3,789 (across all patients)\\
Electrode Type & Intracranial (implanted)\\
Unique Feature & Ambulatory, real-world chronic recording\\
\end{tabular}
\end{center}

This dataset is particularly valuable for:
\begin{itemize}
\item Long-term seizure cycle analysis (circadian + multiday patterns)
\item Real-world seizure forecasting validation
\item Environmental factor studies (e.g., air pollution effects)
\end{itemize}
\subsection{4.3 Performance Metrics Summary}
\label{sec:orgafab65c}

\begin{center}
\begin{tabular}{llllrl}
Study (Year) & Method & Database & Sensitivity & FPR (/h) & SPH\\
\hline
TA-STS-ConvNet (2022) & Triple-attention & CHB-MIT & 96.7\% & 0.072 & -\\
GAMRNN (2023) & GRU+Attention & CHB-MIT & 88.1\% & 0.053 & 5-35 min\\
CNN-LSTM-GRU Hybrid (2025) & Hybrid DL & CHB-MIT & 99.1\% & - & -\\
Multiresolution CNN (2023) & CNN & CHB-MIT & 82\% & 0.058 & -\\
CNN-LSTM Bilinear (2024) & Hybrid Bilinear & CHB-MIT & 98.4\% & 0.02 & -\\
EpiNET GRU-LSTM (2024) & GRU-LSTM & CHB-MIT & 92.5\% & - & 120 min\\
Backwards-Landmark (2024) & SVM + Drift & EPILEPSIAE & 75\% & 1.03 & 10 min\\
Edge DL - LSTM (2021) & LSTM & CHB-MIT & 97.6\% & 0.071 & -\\
Karoly et al. (2020) & Cycle Forecasting & \textbf{NeuroVista} & 69.1\% & - & Multiday\\
Chen et al. (2022) & Air Pollution+EEG & \textbf{NeuroVista} & - & - & -\\
Schroeder et al. (2022) & Pathway Analysis & \textbf{NeuroVista} & - & - & -\\
\end{tabular}
\end{center}
\subsubsection{NeuroVista-Specific Findings}
\label{sec:org99b9249}

From the 6 NeuroVista papers, key methodological insights:

\begin{center}
\begin{tabular}{ll}
Paper & Key Finding\\
\hline
Karoly 2020 (Epilepsia) & Seizure cycles (circadian+multiday) enable 69\% high-risk hit\\
Schroeder 2022 & Seizure pathway ≠ duration; "elasticity" and "semblance"\\
Chen 2022 & CO exposure increases seizure risk (RR: 1.04 per IQR)\\
Goldenholz 2018 & Log-log relationship: mean → variance prediction (94\% acc)\\
DiLorenzo 2019 & First-in-man implantable seizure advisory device development\\
\end{tabular}
\end{center}
\subsection{4.4 Prediction Horizon Analysis}
\label{sec:org4f0184f}

The Seizure Prediction Horizon (SPH) varies significantly across studies:

\begin{center}
\begin{tabular}{lll}
Horizon Range & Description & Studies\\
\hline
5-10 minutes & Short-term prediction, high confidence & Common\\
10-30 minutes & Standard range for clinical intervention & Most studies\\
30-60 minutes & Extended horizon, lower precision & Several\\
60-120 min & Long-term prediction (EpiNET achieved 2 hr) & Emerging\\
\end{tabular}
\end{center}
\subsection{4.5 Deep Learning Methods Comparison}
\label{sec:orga72b878}

\begin{center}
\begin{tabular}{lll}
Method Category & Architectures & Strengths\\
\hline
CNN-based & 1D-CNN, Multiresolution CNN & Spatial/spectral features\\
RNN-based & LSTM, Bi-LSTM, GRU & Temporal sequence modeling\\
Hybrid & CNN-LSTM, CNN-GRU & Combined spatio-temporal\\
Attention-based & GAMRNN, Triple-Attention & Feature importance weighting\\
Graph Networks & sdGCN, GNN & Electrode spatial relations\\
\end{tabular}
\end{center}
\subsection{4.6 Feature Extraction Approaches}
\label{sec:orgc67098f}

\begin{center}
\begin{tabular}{lll}
Feature Type & Examples & Effectiveness\\
\hline
Statistical & Mean, variance, skewness, kurtosis & Baseline performance\\
Spectral & Band power (alpha, beta, delta, theta) & Rhythm-based detection\\
Wavelet & DWT coefficients, energy per band & Time-frequency analysis\\
Entropy & Approximate entropy, sample entropy & Complexity measures\\
Connectivity & Phase-amplitude coupling (PAC), coherence & Inter-channel dynamics\\
Time-domain & Zero-crossing, peak detection & Simple, fast\\
\end{tabular}
\end{center}
\subsection{4.7 Handling Imbalanced Data}
\label{sec:orge185e05}

Seizure events are rare (\textasciitilde{}5\% of recordings), creating severe class imbalance:

\begin{center}
\begin{tabular}{lll}
Technique & Description & Performance Impact\\
\hline
BNNSMOTE & Borderline nearest neighbor oversampling & +15\% accuracy\\
WGAN-GP & Generative adversarial network augment & 91.7\% → 86\% baseline\\
EEGAug & Frequency-band composition augmentation & Improved minority class\\
Class weights & Adjusting loss function weights & Moderate improvement\\
Sliding window overlap & Increase minority samples via overlap & Common approach\\
\end{tabular}
\end{center}
\subsection{4.8 Evaluation Metrics Appropriateness}
\label{sec:org79b6fbc}

For imbalanced seizure data, standard accuracy is misleading:

\begin{center}
\begin{tabular}{lll}
Metric & Appropriateness & Rationale\\
\hline
Accuracy & Low & Biased by majority class\\
Sensitivity (Recall) & High & Critical - must detect seizures\\
Specificity & High & Reduces false alarms\\
False Prediction Rate (FPR) & High & Patient burden measure (/hour)\\
F1-Score & Medium-High & Balances precision/recall\\
AUC-ROC & High & Threshold-independent performance\\
\end{tabular}
\end{center}

\textbf{Best practice:} Report sensitivity + FPR together, as this reflects clinical utility.
\section{5. Summary Diagrams}
\label{sec:org46de788}

\subsection{5.1 Seizure Prediction Pipeline}
\label{sec:org6e6c554}
\begin{center}
\includegraphics[width=.9\linewidth]{./demo_mcp_out/pipeline.png}
\label{org6767ceb}
\end{center}
\subsection{5.2 EEG Brain States}
\label{sec:org8326fc7}
\begin{center}
\includegraphics[width=.9\linewidth]{./demo_mcp_out/states.png}
\label{org4c6fdbf}
\end{center}
\subsection{5.3 Method Evolution Timeline}
\label{sec:orgab20573}
\begin{center}
\includegraphics[width=.9\linewidth]{./demo_mcp_out/evolution.png}
\label{orgfb30617}
\end{center}
\subsection{5.4 Performance vs Complexity Trade-off}
\label{sec:org55514b3}
\begin{center}
\includegraphics[width=.9\linewidth]{./demo_mcp_out/tradeoff.png}
\label{orgc11385b}
\end{center}
\section{6. Trends and Research Gaps}
\label{sec:org3185548}

\subsection{6.1 Current Trends}
\label{sec:orgc67eba2}

\begin{enumerate}
\item \textbf{Hybrid Deep Learning}: CNN-LSTM and attention-based architectures dominate recent literature (2023-2025)
\item \textbf{Patient-Specific Models}: Moving away from general models toward personalized prediction
\item \textbf{Extended Prediction Horizons}: Research pushing from 10-30 min to 60-120 min windows
\item \textbf{Real-time Edge Deployment}: Optimized models for wearable devices and neural implants
\item \textbf{Explainable AI (XAI)}: SHAP and attention visualization for clinical interpretability
\end{enumerate}
\subsection{6.2 Identified Research Gaps}
\label{sec:orgb6da63f}

\begin{center}
\begin{tabular}{lll}
Gap & Current State & Opportunity\\
\hline
Long-term prediction (>2 hr) & Few studies achieve >60 min & Ambulatory warning systems\\
Cross-patient generalization & Patient-specific models dominate & Transfer learning research\\
NeuroVista DL methods & Only 6 papers; mostly non-DL & Apply CNN-LSTM to 521-day data\\
Concept drift adaptation & Only 1-2 studies address & Continuous learning systems\\
Multi-modal fusion & Mostly EEG-only & ECG, EMG, accelerometer fusion\\
False positive burden & FPR 0.05-1.0/h still problematic & Patient quality of life impact\\
Prospective validation & Retrospective dominant & Clinical trial evidence needed\\
Multiday cycle prediction & NeuroVista shows promise & Extend Karoly's cycle forecasting\\
\end{tabular}
\end{center}
\subsection{6.3 NeuroVista Research Gap Analysis}
\label{sec:org055f91d}

Despite being the \textbf{first FDA-approved implantable seizure prediction device trial}, NeuroVista has only 6 indexed papers:
\subsubsection{Why NeuroVista is Underutilized}
\label{sec:orgdde1cea}

\begin{center}
\begin{tabular}{ll}
Factor & Impact\\
\hline
Data access restrictions & Not publicly available like CHB-MIT\\
Small sample size (n=15) & Limits statistical power for DL methods\\
Device-specific focus & Original papers focused on device, not algorithms\\
Geographic concentration & Melbourne-based research group dominates\\
\end{tabular}
\end{center}
\subsubsection{Unique Research Opportunities}
\label{sec:org2946dfe}

\begin{enumerate}
\item \textbf{Longest continuous recording}: 521 days mean vs CHB-MIT's 40 hours mean
\item \textbf{Multiday seizure cycles}: Karoly (2020) showed circadian + multiday patterns
\item \textbf{Environmental factors}: Chen (2022) linked air pollution to seizure risk
\item \textbf{Seizure dynamics}: Schroeder (2022) discovered pathway ≠ duration relationship
\item \textbf{Real-world validation}: Ambulatory data vs hospital-controlled recordings
\end{enumerate}
\subsubsection{Recommended NeuroVista Research Directions}
\label{sec:org00dcc34}

\begin{itemize}
\item Apply modern CNN-LSTM architectures to long-term iEEG data
\item Develop multiday cycle-aware prediction models
\item Study concept drift over months/years of recording
\item Investigate environmental trigger integration
\end{itemize}
\subsection{6.4 Recommendations for Future Research}
\label{sec:orgb1c4172}

\begin{enumerate}
\item \textbf{Address class imbalance systematically}: Standardize WGAN-GP or EEGAug approaches
\item \textbf{Standardize evaluation metrics}: Report sensitivity + FPR + prediction horizon consistently
\item \textbf{Explore concept drift}: EEG patterns change over time - adaptive models needed
\item \textbf{Multi-center validation}: Most studies use CHB-MIT; broader validation required
\item \textbf{Clinical deployment studies}: Bridge gap between research accuracy and clinical utility
\end{enumerate}
\section{7. Conclusion}
\label{sec:org1b27ef6}

This literature review demonstrates the CrossRef Local MCP Server's capability to rapidly search and retrieve relevant academic papers (167M+ indexed works). Key findings:
\subsection{7.1 Seizure Prediction State-of-the-Art}
\label{sec:org70acaf0}
\begin{itemize}
\item \textbf{Best performance}: CNN-LSTM hybrid with attention achieves \textasciitilde{}97\% sensitivity, FPR \textasciitilde{}0.05/h
\item \textbf{Standard prediction horizon}: 10-30 minutes, with emerging 2-hour predictions
\item \textbf{Critical metrics}: Sensitivity and FPR together indicate clinical utility
\item \textbf{Dominant database}: CHB-MIT (23 patients, 916 hours) used in most studies
\end{itemize}
\subsection{7.2 NeuroVista Findings}
\label{sec:orga38f673}
\begin{itemize}
\item \textbf{6 papers found} using simplified "neurovista" query (vs 0 with compound queries)
\item \textbf{Unique dataset}: 15 patients, 521 days mean recording, 3,789 total seizures
\item \textbf{Key insight}: Multiday seizure cycles (Karoly 2020) enable novel forecasting
\item \textbf{Gap}: Modern DL methods (CNN-LSTM) not yet applied to this long-term data
\end{itemize}
\subsection{7.3 Search Strategy Lesson}
\label{sec:org36d38f1}
FTS5 compound queries may fail silently; \textbf{simple single-term queries} are more reliable for named datasets.

\begin{quote}
Generated via CrossRef Local Remote MCP Server
Database: 167,008,748 papers | 1,788,599,072 citations
Search speed: 100-400ms per query (2.83ms for "neurovista")
\end{quote}
\end{document}
